\documentclass[a4paper, 11pt, oneside]{article}

\usepackage[utf8]{inputenc}
\usepackage[T1]{fontenc}
\usepackage[french]{babel}
\usepackage{array}
\usepackage{shortvrb}
\usepackage{listings}
\usepackage[fleqn]{amsmath}
\usepackage{amsfonts}
\usepackage{fullpage}
\usepackage{enumerate}
\usepackage{graphicx}             % import, scale, and rotate graphics
\usepackage{subfigure}            % group figures
\usepackage{alltt}
\usepackage{url}
\usepackage{indentfirst}
\usepackage{eurosym}
\usepackage{listings}
\usepackage{color}
\usepackage[table,xcdraw,dvipsnames]{xcolor}

% Change le nom par défaut des listing
\renewcommand{\lstlistingname}{Extrait de Code}

% Change la police des titres pour convenir à votre seul lecteur
\usepackage{sectsty}
\allsectionsfont{\sffamily\mdseries\upshape}
% Idem pour la table des matière.
\usepackage[nottoc,notlof,notlot]{tocbibind}
\usepackage[titles,subfigure]{tocloft}
\renewcommand{\cftsecfont}{\rmfamily\mdseries\upshape}
\renewcommand{\cftsecpagefont}{\rmfamily\mdseries\upshape}

\definecolor{mygray}{rgb}{0.5,0.5,0.5}
\newcommand{\coms}[1]{\textcolor{MidnightBlue}{#1}}

\lstset{
    language=C, % Utilisation du langage C
    commentstyle={\color{MidnightBlue}}, % Couleur des commentaires
    frame=single, % Entoure le code d'un joli cadre
    rulecolor=\color{black}, % Couleur de la ligne qui forme le cadre
    stringstyle=\color{RawSienna}, % Couleur des chaines de caractères
    numbers=left, % Ajoute une numérotation des lignes à gauche
    numbersep=5pt, % Distance entre les numérots de lignes et le code
    numberstyle=\tiny\color{mygray}, % Couleur des numéros de lignes
    basicstyle=\tt\footnotesize,
    tabsize=3, % Largeur des tabulations par défaut
    keywordstyle=\tt\bf\footnotesize\color{Sepia}, % Style des mots-clés
    extendedchars=true,
    captionpos=b, % sets the caption-position to bottom
    texcl=true, % Commentaires sur une ligne interprétés en Latex
    showstringspaces=false, % Ne montre pas les espace dans les chaines de caractères
    escapeinside={(>}{<)}, % Permet de mettre du latex entre des <( et )>.
    inputencoding=utf8,
    literate=
  {á}{{\'a}}1 {é}{{\'e}}1 {í}{{\'i}}1 {ó}{{\'o}}1 {ú}{{\'u}}1
  {Á}{{\'A}}1 {É}{{\'E}}1 {Í}{{\'I}}1 {Ó}{{\'O}}1 {Ú}{{\'U}}1
  {à}{{\`a}}1 {è}{{\`e}}1 {ì}{{\`i}}1 {ò}{{\`o}}1 {ù}{{\`u}}1
  {À}{{\`A}}1 {È}{{\`E}}1 {Ì}{{\`I}}1 {Ò}{{\`O}}1 {Ù}{{\`U}}1
  {ä}{{\"a}}1 {ë}{{\"e}}1 {ï}{{\"i}}1 {ö}{{\"o}}1 {ü}{{\"u}}1
  {Ä}{{\"A}}1 {Ë}{{\"E}}1 {Ï}{{\"I}}1 {Ö}{{\"O}}1 {Ü}{{\"U}}1
  {â}{{\^a}}1 {ê}{{\^e}}1 {î}{{\^i}}1 {ô}{{\^o}}1 {û}{{\^u}}1
  {Â}{{\^A}}1 {Ê}{{\^E}}1 {Î}{{\^I}}1 {Ô}{{\^O}}1 {Û}{{\^U}}1
  {œ}{{\oe}}1 {Œ}{{\OE}}1 {æ}{{\ae}}1 {Æ}{{\AE}}1 {ß}{{\ss}}1
  {ű}{{\H{u}}}1 {Ű}{{\H{U}}}1 {ő}{{\H{o}}}1 {Ő}{{\H{O}}}1
  {ç}{{\c c}}1 {Ç}{{\c C}}1 {ø}{{\o}}1 {å}{{\r a}}1 {Å}{{\r A}}1
  {€}{{\euro}}1 {£}{{\pounds}}1 {«}{{\guillemotleft}}1
  {»}{{\guillemotright}}1 {ñ}{{\~n}}1 {Ñ}{{\~N}}1 {¿}{{?`}}1
}
\newcommand{\tablemat}{~}

%%%%%%%%%%%%%%%%% TITRE %%%%%%%%%%%%%%%%
% Complétez et décommentez les définitions de macros suivantes :
\newcommand{\intitule}{Rapport Projet 1}
\newcommand{\GrNbr}{23}
\newcommand{\PrenomUN}{Andrew}
\newcommand{\NomUN}{Willems}
\newcommand{\PrenomDEUX}{Pierre}
\newcommand{\NomDEUX}{Lorenzen}
% Décommentez ceci si vous voulez une table des matières :
\renewcommand{\tablemat}{\tableofcontents}

%%%%%%%% ZONE PROTÉGÉE : MODIFIEZ UNE DES DIX PROCHAINES %%%%%%%%
%%%%%%%%            LIGNES POUR PERDRE 2 PTS.            %%%%%%%%
\title{INFO0947: \intitule}
\author{Groupe \GrNbr : \PrenomUN~\textsc{\NomUN}, \PrenomDEUX~\textsc{\NomDEUX}}
\date{}
\begin{document}

\maketitle
\newpage
\tablemat
\newpage
%%%%%%%%%%%%%%%%%%%% FIN DE LA ZONE PROTÉGÉE %%%%%%%%%%%%%%%%%%%%

%%%%%%%%%%%%%%%% RAPPORT %%%%%%%%%%%%%%%
% Écrivez votre rapport ci-dessous.

\section{Description du problème}

Il est demandé de filtrer un tableau d'entiers par rapport à une certaine propriété \textit{p}.

On peut représenter le problème comme suit:

% Please add the following required packages to your document preamble:
% \usepackage[table,xcdraw]{xcolor}
% If you use beamer only pass "xcolor=table" option, i.e. \documentclass[xcolor=table]{beamer}
\begin{table}[!h]
\centering
\begin{tabular}{lllllllllllll}
\multicolumn{1}{l|}{}   & 0                        &                          &                          &                          &                          & \multicolumn{1}{r|}{i}                        &                          &                          &                          &                          &                          & \multicolumn{1}{r|}{N-1}                      \\ \cline{2-13} 
\multicolumn{1}{r|}{T:} & \cellcolor[HTML]{3531FF} & \cellcolor[HTML]{3531FF} & \cellcolor[HTML]{3531FF} & \cellcolor[HTML]{3531FF} & \cellcolor[HTML]{3531FF} & \multicolumn{1}{l|}{\cellcolor[HTML]{3531FF}} & \cellcolor[HTML]{9A0000} & \cellcolor[HTML]{9A0000} & \cellcolor[HTML]{9A0000} & \cellcolor[HTML]{9A0000} & \cellcolor[HTML]{9A0000} & \multicolumn{1}{l|}{\cellcolor[HTML]{9A0000}} \\ \cline{2-13} 
                        &                          &                          &                          &                          &                          &                                               &                          &                          &                          &                          &                          &                                              
\end{tabular}
\end{table}

Avec la zone {\color[HTML]{3531FF}bleu} qui concerne la zone filtrée du tableau(A), la taille de la zone filtrée(B) et tout les éléments qui s'y trouve se trouvait dans le même ordre dans le tableau initiale(c). La zone {\color[HTML]{9A0000} brune} concerne la zone non filtrée ce qui veut dire que la zone est remplie de 0(D).

\section{Spécification des prédicats}

    \subsection{Spécification du prédicat B}

    \begin{enumerate}
        \item Objets Utilisés
            \begin{itemize}
                \item[$\star$] T: Un tableau d'entier initialisé de taille N.
                \item[$\star$] N > 0 ($\in \mathbb{N}$)
                \item[$\star$] p: Une certaine propriété. 
            \end{itemize}

        \item Signature \\
            \textit{TailleZoneFiltree(T,N,p)}

        \item Spécification\\
            \textit{TailleZoneFiltree(T,N,p)}$\equiv \# i, 0\leq i < N, p(T[i])$
    \end{enumerate}

    \subsection{Spécification du prédicat A}

    \begin{enumerate}
        \item Objets Utilisés
            \begin{itemize}
                \item[$\star$] T: Un tableau d'entier initialisé de taille N.
                \item[$\star$] taille: taille de la zone filtrée.
                \item[$\star$] p: Une certaine propriété. 
            \end{itemize}

        \item Signature \\
            \textit{ZoneFiltree(T,p,taille)}

        \item Spécification\\
            \textit{ZoneFiltree(T,p,taille)}$\equiv \forall i, 0\leq i < taille, p(T[i])$
    \end{enumerate}

    \subsection{Spécification du prédicat C}

    \begin{enumerate}
        \item Objets Utilisés
            \begin{itemize}
                \item[$\star$] T$_0$: Le tableau T avant modification.
                \item[$\star$] T: Un tableau d'entier initialisé de taille N.
                \item[$\star$] N > 0 ($\in \mathbb{N}$)
                \item[$\star$] taille: taille de la zone filtrée. 
            \end{itemize}

        \item Signature \\
            \textit{LienTableau(T$_0$,T,N,taille)}

        \item Spécification\\
            \textit{LienTableau(T$_0$,T,N,taille)}$\equiv (\forall i, 1\leq i < taille, (\exists j, 0\leq j < N, T_0[j] = T[i]) \wedge (\exists k, 0\leq k < j, T_0[k] = T[i-1]) )$
    \end{enumerate}

    \subsection{Spécification du prédicat D}

    \begin{enumerate}
        \item Objets Utilisés
            \begin{itemize}
                \item[$\star$] T: Un tableau d'entier initialisé de taille N.
                \item[$\star$] N > 0 ($\in \mathbb{N}$)
                \item[$\star$] taille: taille de la zone filtrée. 
            \end{itemize}

        \item Signature \\
            \textit{ZoneNonFiltree(T,N,taille)}

        \item Spécification\\
            \textit{ZoneNonFiltree(T,N,taille)}$\equiv \forall i, taille < i < N, T[i]=0$
    \end{enumerate}

\end{document}
